% Options for packages loaded elsewhere
\PassOptionsToPackage{unicode}{hyperref}
\PassOptionsToPackage{hyphens}{url}
%
\documentclass[
]{article}
\usepackage{amsmath,amssymb}
\usepackage{lmodern}
\usepackage{iftex}
\ifPDFTeX
  \usepackage[T1]{fontenc}
  \usepackage[utf8]{inputenc}
  \usepackage{textcomp} % provide euro and other symbols
\else % if luatex or xetex
  \usepackage{unicode-math}
  \defaultfontfeatures{Scale=MatchLowercase}
  \defaultfontfeatures[\rmfamily]{Ligatures=TeX,Scale=1}
\fi
% Use upquote if available, for straight quotes in verbatim environments
\IfFileExists{upquote.sty}{\usepackage{upquote}}{}
\IfFileExists{microtype.sty}{% use microtype if available
  \usepackage[]{microtype}
  \UseMicrotypeSet[protrusion]{basicmath} % disable protrusion for tt fonts
}{}
\makeatletter
\@ifundefined{KOMAClassName}{% if non-KOMA class
  \IfFileExists{parskip.sty}{%
    \usepackage{parskip}
  }{% else
    \setlength{\parindent}{0pt}
    \setlength{\parskip}{6pt plus 2pt minus 1pt}}
}{% if KOMA class
  \KOMAoptions{parskip=half}}
\makeatother
\usepackage{xcolor}
\usepackage[margin=1in]{geometry}
\usepackage{graphicx}
\makeatletter
\def\maxwidth{\ifdim\Gin@nat@width>\linewidth\linewidth\else\Gin@nat@width\fi}
\def\maxheight{\ifdim\Gin@nat@height>\textheight\textheight\else\Gin@nat@height\fi}
\makeatother
% Scale images if necessary, so that they will not overflow the page
% margins by default, and it is still possible to overwrite the defaults
% using explicit options in \includegraphics[width, height, ...]{}
\setkeys{Gin}{width=\maxwidth,height=\maxheight,keepaspectratio}
% Set default figure placement to htbp
\makeatletter
\def\fps@figure{htbp}
\makeatother
\setlength{\emergencystretch}{3em} % prevent overfull lines
\providecommand{\tightlist}{%
  \setlength{\itemsep}{0pt}\setlength{\parskip}{0pt}}
\setcounter{secnumdepth}{-\maxdimen} % remove section numbering
\newlength{\cslhangindent}
\setlength{\cslhangindent}{1.5em}
\newlength{\csllabelwidth}
\setlength{\csllabelwidth}{3em}
\newlength{\cslentryspacingunit} % times entry-spacing
\setlength{\cslentryspacingunit}{\parskip}
\newenvironment{CSLReferences}[2] % #1 hanging-ident, #2 entry spacing
 {% don't indent paragraphs
  \setlength{\parindent}{0pt}
  % turn on hanging indent if param 1 is 1
  \ifodd #1
  \let\oldpar\par
  \def\par{\hangindent=\cslhangindent\oldpar}
  \fi
  % set entry spacing
  \setlength{\parskip}{#2\cslentryspacingunit}
 }%
 {}
\usepackage{calc}
\newcommand{\CSLBlock}[1]{#1\hfill\break}
\newcommand{\CSLLeftMargin}[1]{\parbox[t]{\csllabelwidth}{#1}}
\newcommand{\CSLRightInline}[1]{\parbox[t]{\linewidth - \csllabelwidth}{#1}\break}
\newcommand{\CSLIndent}[1]{\hspace{\cslhangindent}#1}
\usepackage{multirow}
\usepackage{multicol}
\usepackage{colortbl}
\usepackage{hhline}
\newlength\Oldarrayrulewidth
\newlength\Oldtabcolsep
\usepackage{longtable}
\usepackage{float}
\usepackage{wrapfig}
\usepackage{array}
\usepackage{hyperref}
\ifLuaTeX
  \usepackage{selnolig}  % disable illegal ligatures
\fi
\IfFileExists{bookmark.sty}{\usepackage{bookmark}}{\usepackage{hyperref}}
\IfFileExists{xurl.sty}{\usepackage{xurl}}{} % add URL line breaks if available
\urlstyle{same} % disable monospaced font for URLs
\hypersetup{
  pdftitle={Heart Disease Project 1},
  pdfauthor={Crystal Sawtelle},
  hidelinks,
  pdfcreator={LaTeX via pandoc}}

\title{Heart Disease Project 1}
\author{Crystal Sawtelle}
\date{STA-486C}

\begin{document}
\maketitle

\hypertarget{introduction}{%
\subsection{Introduction}\label{introduction}}

~~~~~~Heart disease affects hundreds of thousands of people a year and
is the leading cause of death in the United States. There are several
lifestyle choices and medical conditions that can put you at risk for
heart disease including diabetes, unhealthy diet, lack of physical
activity, overweight and obesity, smoking, etc. Another major risk of
having heart disease is genetics CDC (2022). My grandfather died of a
heart attack, my father had quadruple by-pass surgery, and my mother
just recently had a heart attack. I chose the Heart Disease UCI to learn
a little more about the different metrics doctors use to determine if
you are at risk of having heart disease {``Heart {Disease} {UCI}''}
(n.d.). Personally, I know that I am at risk of developing heart disease
and my risk increases as I get older. Analyzing this data will help in
understanding the tests and other symptoms to look out for.

\hypertarget{data-description}{%
\subsection{Data Description}\label{data-description}}

~~~~~~As seen in Table 1 there are 13 variables and 302 observations in
the Heart Disease UCI data set retrieved from Kaggle.com {``Heart
{Disease} {UCI}''} (n.d.). There are five quantitative variables
including age, resting blood pressure, cholesterol, maximum heart rate,
and oldpeak. The minimum age of an individual in this study is 29 years
old and the maximum is 76. The resting blood pressure numbers represents
the systolic pressure (top number) when reading blood pressure. The
range of the observations fall between 94 and 200 with anything above
130 to 140 indicating a cause for concern. The cholesterol levels are
derived from the formula serum = LDL + HDL + 0.2 * triglycerides. Where
LDL stands for low-density lipoprotein and is typically consider the
``bad cholesterol.'' HDL stands fro high-density lipoprotein or the
``good cholesterol.'' HDL absorbs cholesterol in the blood and carries
it back to the liver to be flushed from the body. Triglycerides is a
type of fat in you blood that your body uses for energy. Low-levels of
LDL and/or low-levels HDL with high levels of triglycerides increases
the risk of health problems, like a heart attack. If the serum is
greater than 200 this is typically cause for concern. The maximum heart
rate numbers show a range between 71 and 202. People with a maximum of
over 140 are more at risk of having heart disease. Old peak measures
exercise-induced ST depression versus the heart at rest, a unhealthy
heart will stress more.

\hypertarget{table-1}{%
\subsubsection{Table 1:}\label{table-1}}

\begin{verbatim}
## Warning: fonts used in `flextable` are ignored because the `pdflatex` engine
## is used and not `xelatex` or `lualatex`. You can avoid this warning by using
## the `set_flextable_defaults(fonts_ignore=TRUE)` command or use a compatible
## engine by defining `latex_engine: xelatex` in the YAML header of the R Markdown
## document.
\end{verbatim}

\global\setlength{\Oldarrayrulewidth}{\arrayrulewidth}

\global\setlength{\Oldtabcolsep}{\tabcolsep}

\setlength{\tabcolsep}{0pt}

\renewcommand*{\arraystretch}{1.5}



\providecommand{\ascline}[3]{\noalign{\global\arrayrulewidth #1}\arrayrulecolor[HTML]{#2}\cline{#3}}

\begin{longtable}[c]{|p{1.98in}|p{0.67in}|p{8.62in}}



\ascline{2pt}{666666}{1-3}

\multicolumn{1}{>{\raggedright}p{\dimexpr 1.98in+0\tabcolsep}}{\textcolor[HTML]{000000}{\fontsize{11}{11}\selectfont{Variable}}} & \multicolumn{1}{>{\raggedright}p{\dimexpr 0.67in+0\tabcolsep}}{\textcolor[HTML]{000000}{\fontsize{11}{11}\selectfont{Type}}} & \multicolumn{1}{>{\raggedright}p{\dimexpr 8.62in+0\tabcolsep}}{\textcolor[HTML]{000000}{\fontsize{11}{11}\selectfont{Description}}} \\

\ascline{2pt}{666666}{1-3}\endhead



\multicolumn{1}{>{\raggedright}p{\dimexpr 1.98in+0\tabcolsep}}{\textcolor[HTML]{000000}{\fontsize{11}{11}\selectfont{Age}}} & \multicolumn{1}{>{\raggedright}p{\dimexpr 0.67in+0\tabcolsep}}{\textcolor[HTML]{000000}{\fontsize{11}{11}\selectfont{num}}} & \multicolumn{1}{>{\raggedright}p{\dimexpr 8.62in+0\tabcolsep}}{\textcolor[HTML]{000000}{\fontsize{11}{11}\selectfont{age\ of\ patient}}} \\





\multicolumn{1}{>{\raggedright}p{\dimexpr 1.98in+0\tabcolsep}}{\textcolor[HTML]{000000}{\fontsize{11}{11}\selectfont{Gender}}} & \multicolumn{1}{>{\raggedright}p{\dimexpr 0.67in+0\tabcolsep}}{\textcolor[HTML]{000000}{\fontsize{11}{11}\selectfont{factor}}} & \multicolumn{1}{>{\raggedright}p{\dimexpr 8.62in+0\tabcolsep}}{\textcolor[HTML]{000000}{\fontsize{11}{11}\selectfont{0\ =\ Female}}\textcolor[HTML]{000000}{\fontsize{11}{11}\selectfont{\linebreak }}\textcolor[HTML]{000000}{\fontsize{11}{11}\selectfont{1\ =\ Male}}} \\





\multicolumn{1}{>{\raggedright}p{\dimexpr 1.98in+0\tabcolsep}}{\textcolor[HTML]{000000}{\fontsize{11}{11}\selectfont{Chest\ Pain}}} & \multicolumn{1}{>{\raggedright}p{\dimexpr 0.67in+0\tabcolsep}}{\textcolor[HTML]{000000}{\fontsize{11}{11}\selectfont{factor}}} & \multicolumn{1}{>{\raggedright}p{\dimexpr 8.62in+0\tabcolsep}}{\textcolor[HTML]{000000}{\fontsize{11}{11}\selectfont{chest\ pain:}}\textcolor[HTML]{000000}{\fontsize{11}{11}\selectfont{\linebreak }}\textcolor[HTML]{000000}{\fontsize{11}{11}\selectfont{\ 0\ =\ Typical\ Angina:\ chest\ pain\ related\ to\ decrease\ blood\ supply\ to\ the\ heart}}\textcolor[HTML]{000000}{\fontsize{11}{11}\selectfont{\linebreak }}\textcolor[HTML]{000000}{\fontsize{11}{11}\selectfont{\ 1\ =\ Atypical\ Angina:\ chest\ pain\ not\ related\ to\ the\ heart}}\textcolor[HTML]{000000}{\fontsize{11}{11}\selectfont{\linebreak }}\textcolor[HTML]{000000}{\fontsize{11}{11}\selectfont{\ 2\ =\ non-Anginal\ Pain:\ typical\ esophageal\ spasms\ (not\ heart\ related)}}\textcolor[HTML]{000000}{\fontsize{11}{11}\selectfont{\linebreak }}\textcolor[HTML]{000000}{\fontsize{11}{11}\selectfont{\ 3\ =\ Asymptomatic:\ chest\ pain\ not\ showing\ signs\ of\ heart\ disease}}} \\





\multicolumn{1}{>{\raggedright}p{\dimexpr 1.98in+0\tabcolsep}}{\textcolor[HTML]{000000}{\fontsize{11}{11}\selectfont{Resting\ Blood\ Pressure}}} & \multicolumn{1}{>{\raggedright}p{\dimexpr 0.67in+0\tabcolsep}}{\textcolor[HTML]{000000}{\fontsize{11}{11}\selectfont{num}}} & \multicolumn{1}{>{\raggedright}p{\dimexpr 8.62in+0\tabcolsep}}{\textcolor[HTML]{000000}{\fontsize{11}{11}\selectfont{Resting\ blood\ pressure\ (in\ mm\ Hg\ on\ admission\ to\ the\ hospital)\ anything\ above\ 130-140\ is\ typically\ cause\ for\ concern}}} \\





\multicolumn{1}{>{\raggedright}p{\dimexpr 1.98in+0\tabcolsep}}{\textcolor[HTML]{000000}{\fontsize{11}{11}\selectfont{Cholesterol}}} & \multicolumn{1}{>{\raggedright}p{\dimexpr 0.67in+0\tabcolsep}}{\textcolor[HTML]{000000}{\fontsize{11}{11}\selectfont{num}}} & \multicolumn{1}{>{\raggedright}p{\dimexpr 8.62in+0\tabcolsep}}{\textcolor[HTML]{000000}{\fontsize{11}{11}\selectfont{Cholesterol:\ serum\ cholesterol\ in\ mg/dl\ (milligrams\ per\ deciliter)}}\textcolor[HTML]{000000}{\fontsize{11}{11}\selectfont{\linebreak }}\textcolor[HTML]{000000}{\fontsize{11}{11}\selectfont{\ serum\ =\ LDL\ +\ HDL\ +\ .2*triglycerides}}} \\





\multicolumn{1}{>{\raggedright}p{\dimexpr 1.98in+0\tabcolsep}}{\textcolor[HTML]{000000}{\fontsize{11}{11}\selectfont{Fasting\ Blood\ Sugar}}} & \multicolumn{1}{>{\raggedright}p{\dimexpr 0.67in+0\tabcolsep}}{\textcolor[HTML]{000000}{\fontsize{11}{11}\selectfont{factor}}} & \multicolumn{1}{>{\raggedright}p{\dimexpr 8.62in+0\tabcolsep}}{\textcolor[HTML]{000000}{\fontsize{11}{11}\selectfont{Fasting\ blood\ sugar:\ >\ 120\ mg/dl\ (milligrams\ per\ deciliter)}}\textcolor[HTML]{000000}{\fontsize{11}{11}\selectfont{\linebreak }}\textcolor[HTML]{000000}{\fontsize{11}{11}\selectfont{\ 1\ =\ true}}\textcolor[HTML]{000000}{\fontsize{11}{11}\selectfont{\linebreak }}\textcolor[HTML]{000000}{\fontsize{11}{11}\selectfont{\ 0\ =\ false}}} \\





\multicolumn{1}{>{\raggedright}p{\dimexpr 1.98in+0\tabcolsep}}{\textcolor[HTML]{000000}{\fontsize{11}{11}\selectfont{Resting\ ECG}}} & \multicolumn{1}{>{\raggedright}p{\dimexpr 0.67in+0\tabcolsep}}{\textcolor[HTML]{000000}{\fontsize{11}{11}\selectfont{factor}}} & \multicolumn{1}{>{\raggedright}p{\dimexpr 8.62in+0\tabcolsep}}{\textcolor[HTML]{000000}{\fontsize{11}{11}\selectfont{Resting\ electrocardiographic\ (EKG\ or\ ECG):}}\textcolor[HTML]{000000}{\fontsize{11}{11}\selectfont{\linebreak }}\textcolor[HTML]{000000}{\fontsize{11}{11}\selectfont{\ 0\ =\ nothing\ to\ note}}\textcolor[HTML]{000000}{\fontsize{11}{11}\selectfont{\linebreak }}\textcolor[HTML]{000000}{\fontsize{11}{11}\selectfont{\ 1\ =\ ST-T\ Wave\ abnormality\ (can\ range\ from\ mild\ symptoms\ to\ severe\ problems,\ signals\ non-normal\ heart\ beat)}}\textcolor[HTML]{000000}{\fontsize{11}{11}\selectfont{\linebreak }}\textcolor[HTML]{000000}{\fontsize{11}{11}\selectfont{\ 2\ =\ Possible\ or\ definite\ left\ ventricular\ hypertrophy\ (enlarged\ hearts\ main\ pumping\ chamber)}}} \\





\multicolumn{1}{>{\raggedright}p{\dimexpr 1.98in+0\tabcolsep}}{\textcolor[HTML]{000000}{\fontsize{11}{11}\selectfont{Maximum\ Heart\ Rate}}} & \multicolumn{1}{>{\raggedright}p{\dimexpr 0.67in+0\tabcolsep}}{\textcolor[HTML]{000000}{\fontsize{11}{11}\selectfont{num}}} & \multicolumn{1}{>{\raggedright}p{\dimexpr 8.62in+0\tabcolsep}}{\textcolor[HTML]{000000}{\fontsize{11}{11}\selectfont{Maximum\ heart\ rate\ achieved}}} \\





\multicolumn{1}{>{\raggedright}p{\dimexpr 1.98in+0\tabcolsep}}{\textcolor[HTML]{000000}{\fontsize{11}{11}\selectfont{Exercise\ Induced\ Angina}}} & \multicolumn{1}{>{\raggedright}p{\dimexpr 0.67in+0\tabcolsep}}{\textcolor[HTML]{000000}{\fontsize{11}{11}\selectfont{factor}}} & \multicolumn{1}{>{\raggedright}p{\dimexpr 8.62in+0\tabcolsep}}{\textcolor[HTML]{000000}{\fontsize{11}{11}\selectfont{Exercise\ induced\ angina}}\textcolor[HTML]{000000}{\fontsize{11}{11}\selectfont{\linebreak }}\textcolor[HTML]{000000}{\fontsize{11}{11}\selectfont{\ 1\ =\ true}}\textcolor[HTML]{000000}{\fontsize{11}{11}\selectfont{\linebreak }}\textcolor[HTML]{000000}{\fontsize{11}{11}\selectfont{\ 0\ =\ false}}} \\





\multicolumn{1}{>{\raggedright}p{\dimexpr 1.98in+0\tabcolsep}}{\textcolor[HTML]{000000}{\fontsize{11}{11}\selectfont{Oldpeak}}} & \multicolumn{1}{>{\raggedright}p{\dimexpr 0.67in+0\tabcolsep}}{\textcolor[HTML]{000000}{\fontsize{11}{11}\selectfont{num}}} & \multicolumn{1}{>{\raggedright}p{\dimexpr 8.62in+0\tabcolsep}}{\textcolor[HTML]{000000}{\fontsize{11}{11}\selectfont{ST\ depression\ induced\ by\ exercise\ relative\ to\ rest\ looks\ at\ stress\ of\ heart\ during\ exercise\ (unhealthy\ heart\ will\ stress\ more)}}} \\





\multicolumn{1}{>{\raggedright}p{\dimexpr 1.98in+0\tabcolsep}}{\textcolor[HTML]{000000}{\fontsize{11}{11}\selectfont{Slope}}} & \multicolumn{1}{>{\raggedright}p{\dimexpr 0.67in+0\tabcolsep}}{\textcolor[HTML]{000000}{\fontsize{11}{11}\selectfont{factor}}} & \multicolumn{1}{>{\raggedright}p{\dimexpr 8.62in+0\tabcolsep}}{\textcolor[HTML]{000000}{\fontsize{11}{11}\selectfont{The\ slope\ of\ the\ peak\ exercise\ ST\ segment}}\textcolor[HTML]{000000}{\fontsize{11}{11}\selectfont{\linebreak }}\textcolor[HTML]{000000}{\fontsize{11}{11}\selectfont{\ 0\ =\ Upsloping:\ better\ heart\ rate\ with\ exercise\ (uncommon)}}\textcolor[HTML]{000000}{\fontsize{11}{11}\selectfont{\linebreak }}\textcolor[HTML]{000000}{\fontsize{11}{11}\selectfont{\ 1\ =\ Flatsloping:\ minimal\ change\ (typical\ healthy\ heart)}}\textcolor[HTML]{000000}{\fontsize{11}{11}\selectfont{\linebreak }}\textcolor[HTML]{000000}{\fontsize{11}{11}\selectfont{\ 2\ =\ Downsloping:\ signs\ of\ unhealthy\ heart}}} \\





\multicolumn{1}{>{\raggedright}p{\dimexpr 1.98in+0\tabcolsep}}{\textcolor[HTML]{000000}{\fontsize{11}{11}\selectfont{Fluoroscopy\ Blood\ Flow}}} & \multicolumn{1}{>{\raggedright}p{\dimexpr 0.67in+0\tabcolsep}}{\textcolor[HTML]{000000}{\fontsize{11}{11}\selectfont{factor}}} & \multicolumn{1}{>{\raggedright}p{\dimexpr 8.62in+0\tabcolsep}}{\textcolor[HTML]{000000}{\fontsize{11}{11}\selectfont{Number\ of\ major\ vessels\ colored\ by\ fluoroscopy\ (procedure\ to\ see\ blood\ flow)}}} \\





\multicolumn{1}{>{\raggedright}p{\dimexpr 1.98in+0\tabcolsep}}{\textcolor[HTML]{000000}{\fontsize{11}{11}\selectfont{Heart\ Disease}}} & \multicolumn{1}{>{\raggedright}p{\dimexpr 0.67in+0\tabcolsep}}{\textcolor[HTML]{000000}{\fontsize{11}{11}\selectfont{char}}} & \multicolumn{1}{>{\raggedright}p{\dimexpr 8.62in+0\tabcolsep}}{\textcolor[HTML]{000000}{\fontsize{11}{11}\selectfont{\ 0\ =\ do\ not\ have\ heart\ disease}}\textcolor[HTML]{000000}{\fontsize{11}{11}\selectfont{\linebreak }}\textcolor[HTML]{000000}{\fontsize{11}{11}\selectfont{\ 1\ =\ have\ heart\ disease}}} \\

\ascline{2pt}{666666}{1-3}



\end{longtable}



\arrayrulecolor[HTML]{000000}

\global\setlength{\arrayrulewidth}{\Oldarrayrulewidth}

\global\setlength{\tabcolsep}{\Oldtabcolsep}

\renewcommand*{\arraystretch}{1}

~~~~~~There are nine categorical variables in the data including gender,
chest pain, fasting blood sugar, resting electrocardiograph, exercise
induced angina, slope, fluoroscopic blood flow, and the target variable
Heart Disease. There are roughly twice as many males in this study than
females with 186 males and 92 females represented. Chest pain has four
groups, the first is a typical angina, meaning the subjects have chest
pain related to decrease blood supply to the heart. The second is
atypical angina, which is chest pain that is not related to the heart.
The non-anginal pain is typical esophageal spasms that are not related
to the heart. The last group is asymptomatic chest pain with no signs of
heart disease. Fasting blood sugar is represented by a 1 if the
individual has a blood sugar of over 120 mg/dl and a 0 if the individual
is below 120 mg/dl. Resting electrocardiographic has three groups where
the first group indicates there was nothing to note from the ECG
results. The second group is ST-T wave abnormality which can indicate
mild to severe symptoms of a non-normal heart beat. The last group are
individuals who may have a left ventricular hypertrophy, or an enlarged
of the hearts main pumping chamber. This is the smallest group
consisting of only three individuals, where only one was reported to
have heart disease. Exercise induced angina is also represented by a 1
for individuals who suffer chest pain when exercising and a 0 for
individuals who do not suffer chest pain when exercising. The slope of
the peak exercise ST segment contains three groups. The first group is
upsloping, which indicates a better heart rate with exercise, which is
uncommon. The second group is flatsloping, this indicates minimal change
in heart rate and is considered a typical healthy heart. The last group
is downsloping, this is a sign of an unhealthy heart. Fluoroscopy blood
flow measure the movement of blood in the body, the lower the number the
better the blood flow or no indications of clots in the blood system.
The last categorical variable is whether the individual has heart
disease or not represented by a 1 if they have heart diseasee or 2 if
they do not have heart disease.

\hypertarget{data-evaluation}{%
\subsection{Data Evaluation}\label{data-evaluation}}

~~~~~~Figure 1 provides an exploratory analysis on the categorical
variables against the target, have heart disease No/Yes. These variables
where chosen to compare and find insight into what the data represents.
Graph A in Figure 1 is heart disease by gender where there are 186 males
represented in the data and 92 females. There are roughly twice as many
males in the study than females with 75\% of the females having heart
disease compared to approximately 46\% of the males having heart
disease.

~~~~~~Graph B in Figure 1 is heart disease by chest pain. The largest
group, 128, fall under typical angina, meaning the subjects have chest
pain related to decrease blood supply to the heart. However, there is
only 38 subjects, or 30\% of that group, that show typical angina and
have heart disease. Atypical angina is chest pain that is not related to
the heart. There are 48 subjects who have atypical angina with the
majority, 40 or 93\% reported to have heart disease. The Non-Anginal
Pain is typical esophageal spasms that are not related to the heart.
There are 79 subjects that report non-anginal pain and 64, or 91\%
having heart disease. The last group is Asymptomatic chest pain with no
signs of heart disease. There are 23 subjects in this group with
approximately 70\% having heart disease.

~~~~~~Fasting blood sugar displayed in Graph C Figure 1 has 239
individuals that have a fasting blood sugar below 120 ml/dl and 137 of
them, or 57\% report having heart disease. There are 39 individuals who
have fasting blood sugar above 120 mg/dl and 21, or about 54\% report
having heart disease. Since these figures are roughly close to 50\% for
both True and False, fasting blood sugar may have little to no effect on
the prediction of whether an individual has heart disease or not.

~~~~~~Graph D in Figure 1 is heart disease by resting
electrocardiographic (ECG). There were 67 individuals out of the 133
that reported having heart disease or approximately 50\% where the ECG
indicated there was nothing to note from the results. The second group,
representing the ST-T wave abnormality which can indicate mild to severe
symptoms of a non-normal heart beat, is the largest group consisting of
142 individuals where 90, or approximately 63\% reported having heart
disease. The last group are individuals who may have a left ventricular
hypertrophy, or an enlarged of the hearts main pumping chamber. This is
the smallest group consisting of only three individuals, where only one
was reported to have heart disease.

~~~~~~Heart disease by exercise induced angina in Figure 1 Graph E has
187 individuals that did not have exercise induced angina, out of those
individuals 135 reported having heart disease or roughly 72\%. There
were 91 individuals who reported exercise induced angina where 23, or
25\% reported having heart disease. These results appear to be
incorrect, but the graph is reporting correctly verified by the raw
data.

~~~~~~Figure 1 Graph F is heart disease by the slope of the peak
exercise ST segment. The first group is upsloping, which indicates a
better heart rate with exercise, which is uncommon. There were 19
individuals that had a better heart rate and 9 of them have reported
having heart disease. The second group is flatsloping, this indicates
minimal change in heart rate and is considered a typical healthy heart.
There were 125 individuals that had a flatsloping heart rate and 47, or
approximately 38\% that reported having heart disease. The last and
largest group is downsloping which is a sign of an unhealthy heart
consisting of 134 individuals where 102 reported having heart disease or
approximately 76\%.

~~~~~~Heart disease by blood flow displayed in Graph G of Figure 1
signifies the higher the number the better the blood flow or no
indications of clots in the blood system. The first and largest group
with the lowest blood flow contains 175 individuals, where 130, or 72\%
report having heart disease. The second group contains 65 individuals
where 21 of them report having heart disease or roughly 32\%. The third
group which contains 38 individuals where 7, or 18\% report having heart
disease. The fourth group has 20 individuals, three of which have heart
disease or 15\%. The last group which has the highest blood flow
contains four individuals where three out of the four have heart
disease. This shows a clear trend that the higher the blood flow the
less individuals with heart disease.

\newpage

\includegraphics{HeartDiseaseAnalysis_files/figure-latex/unnamed-chunk-14-1.pdf}

\hypertarget{figure-1}{%
\subsubsection{Figure 1:}\label{figure-1}}

\emph{Exploratory analysis of categorical data, gray is no heart disease
and navy is yes heart disease.} \textbf{Graph A:} \emph{Heart Disease by
Gender.} \textbf{Graph B:} \emph{Heart Disease by Chest Pain; Typical
angina, Atypical angina, Non-Angina Pain, and Asymptomatic.}
\textbf{Graph C:} \emph{Heart Disease by Fasting Blood Sugar; fasting
blood sugar \textgreater{} 120 mg/dl equals true, otherwise false.}
\textbf{Graph D:} \emph{Heart Disease by Resting ECG; Normal, ST-T Wave
Abnormality, Left Ventricular Hypertrophy.} \textbf{Graph E:}
\emph{Heart Disease by Exercise Induced Angina.} \textbf{Graph F:}
\emph{Heart Disease by Slope; Upsloping, Flatsloping, and Downsloping.}
\textbf{Graph G:} \emph{Heart Disease by Blood Flow; higher blood flow
(0-4) less likely to have heart disease.}

\newpage

~~~~~~The box plots in Figure 2 are an exploratory analysis of the
quantitative variables in the Heart Disease data set. The box plots
where chosen to display the statistical summaries of the quantitative
variable and to compare between having heart disease and not having
heart disease. Figure 2 Graph A shows heart disease by age. For
individuals who have not reported having heart disease, the boxplot show
that the median age to be 58 years old, with the majority of the
observations falling between the ages of 52 and 61. The overall range is
between age 39 and 70 with a few outliers below the lower whiskers. For
individuals who have reported having heart disease, the median age is
52, with a overall range between the age of 29 and 76. The majority of
the observations fall between the age of 44 and 59. It is interesting
that the individuals with heart disease appear to be younger in age than
the individuals without heart disease.\\
\hspace*{0.333em}\hspace*{0.333em}\hspace*{0.333em}\hspace*{0.333em}\hspace*{0.333em}\hspace*{0.333em}Graph
B in Figure 2 is heart disease by resting blood pressure. This number
represents the systolic pressure (top number) when reading blood
pressure. There is not much difference in resting blood pressure between
having heart disease and not having heart disease. Both have a median
value of 130 and a quantile 1 value of 120. The quantile 3 value for
individuals without heart disease is 144 and with heart disease is 140.
The majority of resting blood pressure, regardless of heart disease is
between 120 and 140. The overall range for individuals without heart
disease is between 100 and 180 with a few outliers above the upper
whiskers with a max at 200. The overall range for individuals with heart
disease is between 94 and 170 which also have a few outliers above the
upper whiskers with a max at 180. It appears that the systolic blood
pressure value from this data set does not indicate whether an
individual has heart disease or not.\\
\hspace*{0.333em}\hspace*{0.333em}\hspace*{0.333em}\hspace*{0.333em}\hspace*{0.333em}\hspace*{0.333em}Figure
2 Graph C is heart disease by cholesterol level. This number was derived
by serum = LDL + HDL + .2 * triglycerides. If serum calculates to be
greater than 200, there is cause for concern. For individuals reported
to not have heart disease, the median value is 248.5 with the majority
of observations between 215.8 and 281.2. With an overall range between
131 and 353, with an outlier at 409. For individuals who have reported
to have heart disease, the median value is 235.5 with the majority of
observations between 208.2 and 268.8. The overall range for individuals
with heart disease is between 126 and 360, with a few outliers and a max
of 564. The large majority of individuals in the is study regardless of
if they have heart disease or not have a cholesterol level above 200. So
it also appears that the cholesterol level from this dataset does not
indicate whether an individual has heart disease or not.\\
\hspace*{0.333em}\hspace*{0.333em}\hspace*{0.333em}\hspace*{0.333em}\hspace*{0.333em}\hspace*{0.333em}Heart
disease by maximum heart rate is displayed in Figure 2 Graph D. For
individuals who are reported not having heart disease the median maximum
heart rate is 142.5 with the majority of observations falling between
125 and 158. The total overall range is from 88 to 195 with an outlier
at 71. For individuals who reported to have heart disease the median at
161 with the majority falling between 149.2 and 172. The overall range
is from 115 to 202 with a few outliers under the maximum heart rate of
115. The minimum being 96. These box plots indicate that people with
heart disease tend to have a higher maximum heart rate.\\
\hspace*{0.333em}\hspace*{0.333em}\hspace*{0.333em}\hspace*{0.333em}\hspace*{0.333em}\hspace*{0.333em}Graph
E in Figure 2 is heart disease by old peak ST depression. The box plot
for individuals who do not have heart disease has a median value of 1.4
and an overall range between 0 and 4. The majority of the observations
fall between 0.6 and 2.5, with an outlier at 5.6. For individuals with
heart disease the median is at 0.2 with an overall range between 0 and
2.6. The majority of the observations fall between 0 and 1.1 with a few
outliers and a max of 4.2. It appears that the majority of all
observations are at zero and the further away you get from zero the less
likely you will have heart disease.

\hypertarget{analysis-and-results}{%
\subsection{Analysis and Results}\label{analysis-and-results}}

~~~~~~The chi-squared test of independence is used to determine if there
is an association between two categorical variables. The first analysis
was to find if there is any association between each categorical
variable and whether they have heart disease, the predictor variable. My
hypothesis test is as follows:

\[ H_0: \text{ There is no association between the chosen variable and having heart disease} \]
\[ H_a: \text{ There is an association between the chosen variable and having heart disease} \]

\newpage

\includegraphics{HeartDiseaseAnalysis_files/figure-latex/unnamed-chunk-24-1.pdf}

\hypertarget{figure-2}{%
\subsubsection{Figure 2:}\label{figure-2}}

\emph{Exploratory Analysis of quantitative variables. Gold represents
individuals without heart disease and blue represents individuals with
heart disease.} \textbf{Graph A:} \emph{Heart Disease by age.}
\textbf{Graph B:} \emph{Heart Disease by Resting Blood Pressure.}
\textbf{Graph C:} \emph{Heart Disease by Cholesterol Level.}
\textbf{Graph D:} \emph{Heart Disease by Maximum Heart Rate.}
\textbf{Graph E:} \emph{Heart Disease by ST Depression.}

\newpage

\hypertarget{table-2}{%
\subsubsection{Table 2:}\label{table-2}}

\begin{verbatim}
## Warning: fonts used in `flextable` are ignored because the `pdflatex` engine
## is used and not `xelatex` or `lualatex`. You can avoid this warning by using
## the `set_flextable_defaults(fonts_ignore=TRUE)` command or use a compatible
## engine by defining `latex_engine: xelatex` in the YAML header of the R Markdown
## document.
\end{verbatim}

\global\setlength{\Oldarrayrulewidth}{\arrayrulewidth}

\global\setlength{\Oldtabcolsep}{\tabcolsep}

\setlength{\tabcolsep}{0pt}

\renewcommand*{\arraystretch}{1.5}



\providecommand{\ascline}[3]{\noalign{\global\arrayrulewidth #1}\arrayrulecolor[HTML]{#2}\cline{#3}}

\begin{longtable}[c]{|p{3.22in}|p{1.18in}|p{1.09in}}



\ascline{2pt}{666666}{1-3}

\multicolumn{1}{>{\raggedright}p{\dimexpr 3.22in+0\tabcolsep}}{\textcolor[HTML]{000000}{\fontsize{11}{11}\selectfont{Variable}}} & \multicolumn{1}{>{\raggedleft}p{\dimexpr 1.18in+0\tabcolsep}}{\textcolor[HTML]{000000}{\fontsize{11}{11}\selectfont{X\_squared}}} & \multicolumn{1}{>{\raggedleft}p{\dimexpr 1.09in+0\tabcolsep}}{\textcolor[HTML]{000000}{\fontsize{11}{11}\selectfont{P\_Value}}} \\

\ascline{2pt}{666666}{1-3}\endhead



\multicolumn{1}{>{\raggedright}p{\dimexpr 3.22in+0\tabcolsep}}{\textcolor[HTML]{000000}{\fontsize{11}{11}\selectfont{Gender\ vs\ Heart\_Disease}}} & \multicolumn{1}{>{\raggedleft}p{\dimexpr 1.18in+0\tabcolsep}}{\textcolor[HTML]{000000}{\fontsize{11}{11}\selectfont{23.08387946}}} & \multicolumn{1}{>{\raggedleft}p{\dimexpr 1.09in+0\tabcolsep}}{\textcolor[HTML]{000000}{\fontsize{11}{11}\selectfont{0.00000155}}} \\





\multicolumn{1}{>{\raggedright}p{\dimexpr 3.22in+0\tabcolsep}}{\textcolor[HTML]{000000}{\fontsize{11}{11}\selectfont{Chest\ Pain\ vs\ Heart\_Disease}}} & \multicolumn{1}{>{\raggedleft}p{\dimexpr 1.18in+0\tabcolsep}}{\textcolor[HTML]{000000}{\fontsize{11}{11}\selectfont{80.97876151}}} & \multicolumn{1}{>{\raggedleft}p{\dimexpr 1.09in+0\tabcolsep}}{\textcolor[HTML]{000000}{\fontsize{11}{11}\selectfont{0.00000000}}} \\





\multicolumn{1}{>{\raggedright}p{\dimexpr 3.22in+0\tabcolsep}}{\textcolor[HTML]{000000}{\fontsize{11}{11}\selectfont{Fasting\ Blood\ Sugar\ vs\ Heart\_Disease}}} & \multicolumn{1}{>{\raggedleft}p{\dimexpr 1.18in+0\tabcolsep}}{\textcolor[HTML]{000000}{\fontsize{11}{11}\selectfont{0.09240836}}} & \multicolumn{1}{>{\raggedleft}p{\dimexpr 1.09in+0\tabcolsep}}{\textcolor[HTML]{000000}{\fontsize{11}{11}\selectfont{0.76113747}}} \\





\multicolumn{1}{>{\raggedright}p{\dimexpr 3.22in+0\tabcolsep}}{\textcolor[HTML]{000000}{\fontsize{11}{11}\selectfont{Resting\ ECG\ vs\ Heart\_Disease}}} & \multicolumn{1}{>{\raggedleft}p{\dimexpr 1.18in+0\tabcolsep}}{\textcolor[HTML]{000000}{\fontsize{11}{11}\selectfont{9.72968231}}} & \multicolumn{1}{>{\raggedleft}p{\dimexpr 1.09in+0\tabcolsep}}{\textcolor[HTML]{000000}{\fontsize{11}{11}\selectfont{0.00771305}}} \\





\multicolumn{1}{>{\raggedright}p{\dimexpr 3.22in+0\tabcolsep}}{\textcolor[HTML]{000000}{\fontsize{11}{11}\selectfont{Exercise\ Induced\ Angina\ vs\ Heart\_Disease}}} & \multicolumn{1}{>{\raggedleft}p{\dimexpr 1.18in+0\tabcolsep}}{\textcolor[HTML]{000000}{\fontsize{11}{11}\selectfont{55.45620298}}} & \multicolumn{1}{>{\raggedleft}p{\dimexpr 1.09in+0\tabcolsep}}{\textcolor[HTML]{000000}{\fontsize{11}{11}\selectfont{0.00000000}}} \\





\multicolumn{1}{>{\raggedright}p{\dimexpr 3.22in+0\tabcolsep}}{\textcolor[HTML]{000000}{\fontsize{11}{11}\selectfont{Slope\ vs\ Heart\_Disease}}} & \multicolumn{1}{>{\raggedleft}p{\dimexpr 1.18in+0\tabcolsep}}{\textcolor[HTML]{000000}{\fontsize{11}{11}\selectfont{46.88947660}}} & \multicolumn{1}{>{\raggedleft}p{\dimexpr 1.09in+0\tabcolsep}}{\textcolor[HTML]{000000}{\fontsize{11}{11}\selectfont{0.00000000}}} \\





\multicolumn{1}{>{\raggedright}p{\dimexpr 3.22in+0\tabcolsep}}{\textcolor[HTML]{000000}{\fontsize{11}{11}\selectfont{Blood\ Flow\ vs\ Heart\_Disease}}} & \multicolumn{1}{>{\raggedleft}p{\dimexpr 1.18in+0\tabcolsep}}{\textcolor[HTML]{000000}{\fontsize{11}{11}\selectfont{73.68984583}}} & \multicolumn{1}{>{\raggedleft}p{\dimexpr 1.09in+0\tabcolsep}}{\textcolor[HTML]{000000}{\fontsize{11}{11}\selectfont{0.00000000}}} \\

\ascline{2pt}{666666}{1-3}



\end{longtable}



\arrayrulecolor[HTML]{000000}

\global\setlength{\arrayrulewidth}{\Oldarrayrulewidth}

\global\setlength{\tabcolsep}{\Oldtabcolsep}

\renewcommand*{\arraystretch}{1}

~~~~~~As seen in Table 2 the X\_squared test statistic for gender, chest
pain, resting ECG, exercise induced angina, slope and blood flow are
mostly higher numbers and using \(\alpha \le 0.05\) the p-values are
below 0.05. For these variables we would reject the null hypothesis,
concluding there is evidence that there is an association between these
specific variable and having heart disease. For fasting blood sugar we
would fail to reject the null hypothesis because the X\_squared value is
low and the p-value is approximately 0.76, which is greater than 0.05.
There is no evidence there is an association between fasting blood sugar
and having heart disease. This chi-squared test of independence verified
what the visualizations in Figure 1 displayed, that fasting blood sugar
is not a significant variable when determining if an individual has
heart disease.

~~~~~~The second analysis was to find if there is any association
between each categorical variable and the slope, the slope of the peak
exercise ST segment. I chose the slope to see if there was any
association because the data showed a clear distinction that people with
downsloping ST segment typically have heart disease. My hypothesis test
is as follows:

\[ H_0: \text{ There is no association between the chosen variable and the slope} \]
\[ H_a: \text{ There is an association between the chosen variable and the slope} \]

\hypertarget{table-3}{%
\subsubsection{Table 3:}\label{table-3}}

\begin{verbatim}
## Warning: fonts used in `flextable` are ignored because the `pdflatex` engine
## is used and not `xelatex` or `lualatex`. You can avoid this warning by using
## the `set_flextable_defaults(fonts_ignore=TRUE)` command or use a compatible
## engine by defining `latex_engine: xelatex` in the YAML header of the R Markdown
## document.
\end{verbatim}

\global\setlength{\Oldarrayrulewidth}{\arrayrulewidth}

\global\setlength{\Oldtabcolsep}{\tabcolsep}

\setlength{\tabcolsep}{0pt}

\renewcommand*{\arraystretch}{1.5}



\providecommand{\ascline}[3]{\noalign{\global\arrayrulewidth #1}\arrayrulecolor[HTML]{#2}\cline{#3}}

\begin{longtable}[c]{|p{2.60in}|p{1.09in}|p{1.09in}}



\ascline{2pt}{666666}{1-3}

\multicolumn{1}{>{\raggedright}p{\dimexpr 2.6in+0\tabcolsep}}{\textcolor[HTML]{000000}{\fontsize{11}{11}\selectfont{Variable}}} & \multicolumn{1}{>{\raggedleft}p{\dimexpr 1.09in+0\tabcolsep}}{\textcolor[HTML]{000000}{\fontsize{11}{11}\selectfont{X\_squared}}} & \multicolumn{1}{>{\raggedleft}p{\dimexpr 1.09in+0\tabcolsep}}{\textcolor[HTML]{000000}{\fontsize{11}{11}\selectfont{P\_Value}}} \\

\ascline{2pt}{666666}{1-3}\endhead



\multicolumn{1}{>{\raggedright}p{\dimexpr 2.6in+0\tabcolsep}}{\textcolor[HTML]{000000}{\fontsize{11}{11}\selectfont{Gender\ vs\ Slope}}} & \multicolumn{1}{>{\raggedleft}p{\dimexpr 1.09in+0\tabcolsep}}{\textcolor[HTML]{000000}{\fontsize{11}{11}\selectfont{0.6700921}}} & \multicolumn{1}{>{\raggedleft}p{\dimexpr 1.09in+0\tabcolsep}}{\textcolor[HTML]{000000}{\fontsize{11}{11}\selectfont{0.71530514}}} \\





\multicolumn{1}{>{\raggedright}p{\dimexpr 2.6in+0\tabcolsep}}{\textcolor[HTML]{000000}{\fontsize{11}{11}\selectfont{Chest\ Pain\ vs\ Slope}}} & \multicolumn{1}{>{\raggedleft}p{\dimexpr 1.09in+0\tabcolsep}}{\textcolor[HTML]{000000}{\fontsize{11}{11}\selectfont{27.3926971}}} & \multicolumn{1}{>{\raggedleft}p{\dimexpr 1.09in+0\tabcolsep}}{\textcolor[HTML]{000000}{\fontsize{11}{11}\selectfont{0.00012222}}} \\





\multicolumn{1}{>{\raggedright}p{\dimexpr 2.6in+0\tabcolsep}}{\textcolor[HTML]{000000}{\fontsize{11}{11}\selectfont{Fasting\ Blood\ Sugar\ vs\ Slope}}} & \multicolumn{1}{>{\raggedleft}p{\dimexpr 1.09in+0\tabcolsep}}{\textcolor[HTML]{000000}{\fontsize{11}{11}\selectfont{3.3472109}}} & \multicolumn{1}{>{\raggedleft}p{\dimexpr 1.09in+0\tabcolsep}}{\textcolor[HTML]{000000}{\fontsize{11}{11}\selectfont{0.18756957}}} \\





\multicolumn{1}{>{\raggedright}p{\dimexpr 2.6in+0\tabcolsep}}{\textcolor[HTML]{000000}{\fontsize{11}{11}\selectfont{Resting\ ECG\ vs\ Slope}}} & \multicolumn{1}{>{\raggedleft}p{\dimexpr 1.09in+0\tabcolsep}}{\textcolor[HTML]{000000}{\fontsize{11}{11}\selectfont{10.6435610}}} & \multicolumn{1}{>{\raggedleft}p{\dimexpr 1.09in+0\tabcolsep}}{\textcolor[HTML]{000000}{\fontsize{11}{11}\selectfont{0.03087589}}} \\





\multicolumn{1}{>{\raggedright}p{\dimexpr 2.6in+0\tabcolsep}}{\textcolor[HTML]{000000}{\fontsize{11}{11}\selectfont{Exercise\ Induced\ Angina\ vs\ Slope}}} & \multicolumn{1}{>{\raggedleft}p{\dimexpr 1.09in+0\tabcolsep}}{\textcolor[HTML]{000000}{\fontsize{11}{11}\selectfont{24.7556872}}} & \multicolumn{1}{>{\raggedleft}p{\dimexpr 1.09in+0\tabcolsep}}{\textcolor[HTML]{000000}{\fontsize{11}{11}\selectfont{0.00000421}}} \\





\multicolumn{1}{>{\raggedright}p{\dimexpr 2.6in+0\tabcolsep}}{\textcolor[HTML]{000000}{\fontsize{11}{11}\selectfont{Blood\ Flow\ vs\ Slope}}} & \multicolumn{1}{>{\raggedleft}p{\dimexpr 1.09in+0\tabcolsep}}{\textcolor[HTML]{000000}{\fontsize{11}{11}\selectfont{11.2115315}}} & \multicolumn{1}{>{\raggedleft}p{\dimexpr 1.09in+0\tabcolsep}}{\textcolor[HTML]{000000}{\fontsize{11}{11}\selectfont{0.18999920}}} \\

\ascline{2pt}{666666}{1-3}



\end{longtable}



\arrayrulecolor[HTML]{000000}

\global\setlength{\arrayrulewidth}{\Oldarrayrulewidth}

\global\setlength{\tabcolsep}{\Oldtabcolsep}

\renewcommand*{\arraystretch}{1}

~~~~~~What can be seen in Table 3 is that gender has a p-value of
\(\gt 0.71\) which is greater than \(\alpha = 0.05\), we would fail to
reject the null hypothesis, there is no evidence there is an association
between gender and the slope. The same can be said for both fasting
blood sugar and blood flow, both have a p-value equal to
\(\approx .19\). Again showing not evidence there is an association
between the two variables and the slope. The three remaining variables
are chest pain, resting ECG and exercise induced angina. All of which
have p-values less than 0.05. In this case we would reject the null
hypothesis, there is evidence there is an association between these
three variables and the slope.

\hypertarget{discussion-and-conclusion}{%
\subsection{Discussion and Conclusion}\label{discussion-and-conclusion}}

~~~~~~In reviewing this data, I have learned there are a number of
warning signs that I personally can look out for that could be a sign of
heart disease. Although most of these metrics can only be determined by
a doctors visit and necessary tests, chest pain and exercise induced
angina can be clear signs without the need of a doctor. If I had more
time to analyze this data further, I would like to pull data directly
from the source and not through the kaggle website. The There are a lot
of other things that this data left out that could help predict heart
disease, like weight, exercise and other lifestyle parameters.
Additionally, I would have done further analysis on the quantitative
variables and produce a logistic regression model with test and training
sets to better predict heart disease.

\hypertarget{appendix}{%
\subsection{Appendix}\label{appendix}}

\href{https://github.com/dutchess3030/HeartDisease/blob/main/doc/HeartDiseaseAnalysis.Rmd}{Github
code}

\hypertarget{refs}{}
\begin{CSLReferences}{1}{0}
\leavevmode\vadjust pre{\hypertarget{ref-cdc_heart_2022}{}}%
CDC. 2022. {``Heart {Disease} {Facts} {\textbar} Cdc.gov.''}
\emph{Centers for Disease Control and Prevention}.
\url{https://www.cdc.gov/heartdisease/facts.htm}.

\leavevmode\vadjust pre{\hypertarget{ref-noauthor_heart_nodate}{}}%
{``Heart {Disease} {UCI}.''} n.d. Accessed February 23, 2023.
\url{https://www.kaggle.com/datasets/hartman/heart-disease-uci}.

\end{CSLReferences}

\end{document}
